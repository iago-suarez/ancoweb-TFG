\chapter{Resultados e Conclusións}

Pódese concluír por tanto, que é posible crear unha web para a análise de vídeo que teña a 
capacidade de empregarse como ferramenta de investigación na busca áxil de novos métodos, tanto
de seguimento de obxectos como de análise de alto nivel.

O resultado ao que se chegou dista moito de ser o idóneo, pero sen ser perfecto é un avance 
significativo capaz de amosar as capacidades deste tipo de sistemas, sentando un precedente para
que nun futuro próximo se siga a mellorar esta aplicación de cara a dar visibilidade a estes 
traballos de investigación que moitas veces teñen tan complicado obter visibilidade que merecen.

\chapter{Liñas Futuras}
\label{sec:linhasFuturas}
    En canto ás liñas futuras da aplicación, estas están compostas en parte por incidencias do 
    Product Backlog que por falta de tempo ou por ter unha baixa prioridade quedaron como tarefas 
    abertas no sitema de Xestión de Incidencias, e tamén en parte por outras funcionalidades que se
    consideran de interese no futuro pero que en ningún momento chegaron a entrar no Product Backlog
    polas súas esixencias en tempo.
    
    \section{Incidencias rexistradas como abertas ou re-abertas en YouTrack}
    
    \begin{itemize}
     \item \textbf{Cambiar as sentencias Popen shell='TRUE':}     
        Popen é unha clase empregada dende o código python do servidor para executar dende a 
        terminal unha acción determinada, pero ao executar co parámetro shell=TRUE, a información
        de se se produciu algún erro perdese. Deberíase empregar co modo shell=TRUE inda que isto
        require de varias adaptacións no código.
     \item \textbf{Internacionalización:}
        Estaría ben traducir a aplicación a Español e Galego.
     \item \textbf{Cancelar Subida:}
        Actualmente existe un bug que non permite cancela unha subida mentres esta está en proceso,
        o problema está asociado ao uso de thread's debido ao acceso a Base de datos.
     \item \textbf{Modificar o algoritmo de actualización da barra de progreso:}
        En un futuro, deberíase modificar o algoritmo de progreso de unha notificación de subida para intentar
        minimizar o número de consultas AJAX ao servidor, de xeito que se incremente o retardo entre peticións
        cando o avance de barra de progreso sexa lento e se acelere este cando o avance sexa veloz.
    \item \textbf{Mostrar Potencial:}
        Unha funcionalidade futura sería a de mostrar en cada momento cales son os puntos polo que 
        pasan mais obxectos, isto está almacenado no sistema de recoñecemento como unha matriz que 
        contén o potencial de cada punto da imaxe, e a tarefa consistiría en amosar estes potenciais
        nunha capa de información superposta ao vídeo.
    \item \textbf{Mostrar velocidade:}
        Mostrar a velocidade de cada un dos obxectos en escena así como permitir filtrar por un 
        límite de velocidade ao igual que se fai co comportamento anómalo.
    \item \textbf{Cobertura dos test da capa web:}
        Esta tarefa foi desprazada a unha rama a parte, xa que o esforzo para a súa configuración é
        cuantioso e dada a súa baixa prioridade non se chegou a completar. A solución pasa por 
        empregar algunha ferramenta como blanketJS que recolla a cobertura en phantoJS, e logo 
        enviar esta ao servidor de coveralls. 
    \item \textbf{Error ao subir dous vídeos simultaneamente}
        Revisar esta casuística na que se soben dous vídeos simultaneamente pois en certos momentos
        parece ter un comportamento erróneo.
    \item \textbf{Illar Tracking da análise de alto nivel}
        De cara a facilitar a investigación estaría ben poder realizar solo o análise de alto nivel 
        proporcionando a lista de deteccións, ou simplemente empregar unha análise de baixo nivel 
        para detectar obxectos.
    \item \textbf{Engadir compatibilidade con IE}
        Actualmente a aplicación non é compatible con Internet Explorer, en un futuro estaría ben 
        soportar este navegador.
    \end{itemize}

    
    \section{Outras funcionalidades de interese}
    
        \subsection{Vídeo en Directo}
            De cara a un futuro a funcionalidade mais desexada é a de facer o análise en directo 
            empregando o vídeo procedente dunha cámara. Pero isto prantexa unha serie de retos 
            importantes como é a sincronización do vídeo co tempo de análise ou o envío dos datos desta 
            análise á capa web, xa que isto rompe co modelo HTTP de peticións a un servidor.
            
            A solución para o problema da sincronización entre o vídeo e o tracking, pasa por retardar o
            envío do vídeo unha determinada cantidade de tempo, suficiente como para que se procese e se
            faga a análise.
            
            Non obstante como o tempo de análise non se pode prever de forma exacta, habería
            necesariamente que facer axustes para que cando a análise fose mais veloz que a reprodución
            do vídeo, a capa web mantese un buffer cos resultados recibidos e os amosase no momento 
            correcto. E por outra parte se o vídeo evolucionase mais rápido que a análise, o lado 
            servidor debería reducir o número de fotogramas a analizar descartando por exemplo un de
            cada dous ou de cada tres fotogramas.
            
            Para o problema do envío dos datos á capa web, o mais lóxico podería ser empregar unha 
            modificación do protocolo RTSP (Real-Time Streaming Protocol) visto na sección 
            \ref{sec:streaming}, de xeito que por un lado se envíe o vídeo en directo, e por outra parte
            toda a información precisa para a súa reprodución mais os datos en XML sobre as deteccións.
        
        \subsection{Panel de Control}
            Se se chegase a desenvolver un módulo que permita streaming de vídeo en directo sería de
            gran interese ter un panel no que se poidan visualizar as imaxes procedentes de 
            distintas cámaras simultaneamente. Inda que isto tamén podería facerse integrando a 
            aplicación con algunha outra aplicación de vídeo-vixilancia open-source como as citadas 
            no apartado \ref{sec:video-vixilancia-libre}
  
  
  
  
  
  
  
  
  
  
