\chapter{Fundamentos Teóricos e Conceptos Previos}

O desenvolvemento dunha aplicación web non é algo trivial e mais se temos en conta todas
as peculiaridades que este proxecto contén. Para a súa comprensión é preciso coñecer unha
serie de conceptos teóricos que se expoñen a continuación:

\section{Arquitectura web}
	Neste proxecto séguese unha Arquitectura Web baseada no modelo cliente-servidor, que 
	consiste nun lado servidor que distribúe os recursos como poden ser o contido multimedia
	(vídeos, imaxes, etc) ou as páxinas web ao outro lado, o cliente, que típicamente corre 
	nun navegador web interpretando as páxinas html e o código javascript asociado a estas.
	
	No noso caso a parte servidor estará dividida en dúas compoñentes claramente diferenciadas,
	o sistema para a análise do comportamento, e a aplicación web que permitirá o acceso a este, 
	os fundamentos de ámbalas dúas partes explícanse a continuación.
	
\section{Análise do comportamento}
	É un dos campos de investigación mas activos hoxe en día. A idea principal na que se 
	centran estes sistemas como o que nos ocupa é a de detectar calquera acción levada a 
	cabo polos obxectos involucrados nunha escena de vídeo. Un obxecto é calquera cousa 
	que debe ser seguida, polo que dependendo do tipo de problema estes obxectos poden 
	ser dunha natureza ou doutra.
	O tipo de accións a detectar tamén depende da tipoloxía do sistema, xa poden ser
	comportamentos individuais( camiñar, correr, loitar...) ou grupais (reunirse, abandonar
	un grupo de persoas...).
	
	Tanto neste proxecto como nos sistemas para a análise do comportamento en xeral, 
	pódense discernir tres tarefas importantes que colaboran entre si\cite{brais-thesis}:
	
	\begin{itemize}
	
		\item{\textbf{Detección de Obxectos:}}\label{cap:DeteccionObxetos} Partindo dunha secuencia de vídeo como 
		entrada obtéñense os distintos obxectos que aparecen en cada fotograma da escena.
		Para este fin empréganse técnicas de visión por computador.
		
		%Aqui podríame extender falando da Substracción de Fondo, o Fluxo Óptico e os Sistemas de alto Nivel
		
		\item{\textbf{Seguimento de Obxectos:}} A partires da información obtida na detección, asígnanselle 
		identificadores a cada obxecto detectado no vídeo, agrupando se procede distintos
		obxectos baixo o mesmo identificador en caso de considerarse que estes obxectos forman
		parte de un grupo ou unha mesma detección.
		
		% Aquí podría falar das aproximacións por apariencia, filtro de Kalman, filtro de Partículas ..
		
		\item{\textbf{Análise do comportamento de Alto Nivel:}} Unha vez obtida a información dos 
		dous pasos anteriores pódese catalogar o comportamento de cada detección empregando
		técnicas de recoñecemento de patróns.
	
	\end{itemize}	
	
	Os resultados mais destacables destas técnicas cos que a aplicación terá que traballar serán:
	\begin{itemize}
		\item A lista de obxectos detectados para cada un dos fotogramas e a súa posición neles
		\item A traxectoria de cada un dos obxectos detectados
		\item O grao de anormalidade da traxectoria seguida por un obxecto en cada un dos fotogramas
		\item A velocidade de un obxecto determinado.
	\end{itemize}
	
	Estas tres tarefas requiren dunha serie de cálculos matemáticos baseados en técnicas moi diversas,
	para realizar toda esta serie de cálculos, empregase algunha biblioteca de código que simplifique 
	o traballo a realizar, e neste caso, esta biblioteca é OpenCV.
	
\section{Programación Web}

	A programación web de aplicacións de carácter empresarial require do coñecemento da rede, ademais
	do de unha serie de ferramentas e estratexias para chegar a un deseño sostible e de calidade.
	
	A arquitectura clásica das aplicacións web pódese ver no gráfico \ref{fig:ArquitecturaAppWeb}
	
	\begin{figure}[htp]
	\begin{center}
		\includegraphics[scale=0.35]{figures/ArquitecturaAppWeb.png}
		\caption{Clásica arquitectura dunha aplicación web empresarial}
	\label{fig:ArquitecturaAppWeb}
	\end{center}
	\end{figure}

	As técnicas e estratexias mais importantes á hora de construír unha aplicación web relátanse nos
	puntos subseguistes:
	
	\subsection {Desenvolvemento Áxil}
		Para reducir custos e poder proporcionar solucións rápidas é preciso que as 
		aplicacións web's de caracter empresarial se leven a cabo en pouco tempo e con
		bos principios de enxeñaría, a isto contribúen en gran medida as tecnoloxías 
		que se amosan a continuación.
	
	\subsection{Soporte para transaccións}
		Unha transacción nun Sistema Xestor de Base de Datos (SGBD) é un conxunto de ordes que
		se executan formando unha unidade de traballo, de forma invisible e atómica. 
		As transaccións cobran gran importancia nas aplicacións web debido á inestabilidade 
		da rede e á concorrencia dos distintos clientes conectados, polo que é axuda a un 
		desenvolvemento moi áxil que a tecnoloxía traia a súa xestión integrada.
	
	\subsection{Object-Relational Mapping (ORM)}
		Os mapeado obxecto-relación é unha das técnicas de programación que mais velocidade imprimen
		na construcción de webs, xa que converte os datos dunha linguaxe Orientada a Obxectos (OO) a 
		datos de un sistema relacional no que son persistidos e viceversa, aforrando ao programador
		o traballo de ter que programar o código para esta tarefa. É desexable pois que a tecnoloxía
		a empregar dispoña de un mapeador obxecto-relacional ben integrado, algúns exemplos disto poden
		ser: a combinación Java+Maven+Hibernate, o EF(Entity Framework) de Microsoft ou os Models de Django.
	
	\subsection{Xestión de Layout}
		Tamén resulta moi practico dispoñer dunha linguaxe de prantilla que permitan xerar contido
		html ben estruturado dinamicamente. Algúns exemplos son o Sistema de Templates de Django, o 
		Sistema JSP de Spring, a libraría Thymeleaf ou os compoñentes de ASP.NET. Todos eles axudan 
		a xerar contido HTML de xeito sinxelo e escalable.
		
    \subsection{AJAX}
		
	\subsection{Outras cuestións da web}
		A maiores existe toda unha gama de outras funcionalidades que cobran importancia cando deseñamos
		e construímos unha web como o manexo de erros nos formularios, internacionalización (i18n), 
		visualización de grande cantidades de datos(en listas ou táboas), seguridade...
		
		
\section{O Vídeo}

    \subsection{Formato de Vídeo}
    
    \subsection{Codec's}

    \subsection{Streaming de Vídeo}
        Dado que este proxecto está centrado no tratamento de vídeo, é de especial importancia ver 
        de que xeitos podemos distribuílo e reproducilo a través da rede. A estes efectos existen dúas
        grandes alternativas que varían en canto ao seu grao de escalabilidade, dificultade de 
        implementación, e calidade final do servizo:
        
        \subsection{Pseudo Streaming ou Descarga Progresiva}
            Consiste na descarga do vídeo por fragmentos, típicamente empregando o protocolo HTTP. 
            Neste formato, o reprodutor vai acumulando fragmentos de vídeo ata obter os precisos como 
            para comezar a reprodución, mais se o ancho de banda fose insuficiente, o vídeo remataría
            por pararse. Este sistema é o empregado por servizos como YouTube, Vimeo, DailyMotion...
            
            Será a opción empregada por motivos de simplicidade, mais compre explicar tamén o verdadeiro 
            Streaming, xa que é a diferenza do pseudo-streaming pode ser empregado para a emisión de
            contido en directo como o dunha cámara de seguridade.
            
        \subsection{Streaming}
            O verdadeiro streaming ( do inglés True Streaming) consiste na emisión en directo do 
            contido multimedia a través da rede, que o reprodutor reproduce no momento que recibe.
            Este outro xeito de distribuír vídeo, apoiase en axustar a calidade do vídeo ao ancho de
            banda do que dispón o cliente, evitando así interrupcións na reprodución.
            
            O protocolo mais destacable á hora de empregar este tipo de streaming é RTSP (Real-Time
            Streaming Protocol) que operando a nivel de aplicación permite controlar un ou varios fluxos 
            sincronizados de contido multimedia.
            
            Por unha parte RTSP soe empregar o Real-Time Transport Protocol (RTP) sobre UDP(User 
            Datagram Protocol) para o transporte de contido multimedia, maximizando así o emprego 
            da rede pero sen garantir un mínimo na calidade do servizo.
            
            E por outra parte RTSP emprega o Real-time Control Protocol (RTCP) sobre TCP(Transmission 
            Control Protocol) para a transmisión periódica de paquetes de control da sesión, o
            diagnóstico de fallos e o control de la calidade da transmisión.
            
            \begin{figure}[htp]
            \begin{center}
                \includegraphics[scale=0.6]{figures/RTPS-diagram.png}
                \caption{Diagrama conexión RTPS}
            \label{fig:RTPS-diagram}
            \end{center}
            \end{figure}
            
            RTSP asemellase a HTTP no formato das peticións/repostas e na sintaxe, pero dispoñendo 
            dun estado que permite tanto a clientes como a servidores facer peticións.

            Tamén existen outros protocolos propietarios como MMS (Microsoft Media Server)ou RTMP 
            (Real-Time Messaging Protocol) e RTMFP (Real-Time Media Flow Protocol) de Adobe.
  
\chapter{Análise de antecedentes e alternativas}
	Se trata de realizar un estudio de alternativas o “estado del arte” o un análisis comparativo
	de alternativas.

	Se exponen las diferentes alternativas que se han evaluado o que se consideran de interés, a
	lo realizado en el proyecto. Fundamentalmente se trata de otras herramientas existentes 
	que realizan algo similar, sean o no comerciales, o de prototipos de investigación relacionados,
	o de estudios que tratan aspectos similares.

	Buscar por internet produtos que fagan algo similar...