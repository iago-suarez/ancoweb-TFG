\chapter{Estudo das tecnoloxías para a creación de aplicacións web baseadas en vídeo}

	Para a elaboración de este traballo de fin de grao, é preciso seleccionar unha serie de
	tecnoloxías, vamos a empregar os seguintes criterios para poder comparalas entre si e 
	escoller entre elas a que mellor se adecúan ás necesidades do proxecto.
	
\begin{itemize}

	\item {\textbf{Plataforma e Portabilidade\\}}
		Segundo as especificacións iniciais do proxecto, e de cara a facilitar a 
		implementación deste, as ferramentas empregadas deben de executarse baixo os
		Sistemas Operativos baseados en Linux.

	\item {\textbf{Compatibilidade co algoritmo de Procesado de Vídeo\\}}
		Dado que o algoritmo que procesa o vídeo foi parcialmente implementado en C++,
		a tecnoloxía que se empregue para o desenvolvemento da parte web debe dispoñer da
		máxima compatibilidade con esta linguaxe de programación.

	\item {\textbf{Desenvolvemento Áxil\\}}
		A tecnoloxía que se seleccione ten que minimizar o custe en tempo e esforzo da 
		implementación, por este motivo valorarase positivamente que dispoña de IDE's axeitados,
		facilidades de acceso a BD(Base de Datos),se é posible ORM(Object-Relational Mapping) 
		integrado, recarga en quente... En xeral todo aquelo que permita axilidade e flexibilidade.

\end{itemize}

Posto que empregamos unha arquitectura baseada no modelo cliente-servidor, temos que determinar 
en que tecnoloxías vamos a construír por un lado o Servidor ou Back-End, e por outra parte o 
Cliente ou Front-End que se executará nun navegador.  

\section{Back-End}
	En canto ás tecnoloxías para a elaboración do Back-End, hai que ter en conta que procesará
	os datos proporcionados polo Front-End, atendendo as súas peticións, e xestionando o modelo de 
	datos e os procesos implicados na aplicación. A maiores neste caso en concreto, o Back-End será
	o encargado de interactuar directamente co sistema de Análise de Vídeo.\\
	
	As tecnoloxías estudadas para esta parte do sistema son as seguintes:

	\subsection{Java}
  	Java é unha das linguaxes mais empregadas actualmente. Ademais existen diversos
  	frameworks web como Tapestry ou SpringMVC e en canto as BD Hibernate, que facilitan o 
  	seu uso, mais é preciso integralos xa que non forman parte da plataforma en si. A súas 
  	posibilidades de integración con c++ son altas grazas á interface JNI(Java Native 
  	Interface), pero a súa configuración pódese volver tediosa. Un dos proxectos estudados
  	para ver o seu funcionamento é Red5 \cite{red5-github-url}.
  	
	\subsection{C\#}
	Esta linguaxe en combinación con .NET resulta unha combinación bastante áxil de cara
	á programación web, integrando na propia plataforma un deseñador Web e un ORM moi 
	intuitivo. O gran problema polo que se descartou este entorno foi polo seu baixo grao
	de compatibilidade cos sistemas operativos Linux.
	
	\subsection{C - C++}
	C++ presenta como era de esperar a maior compatibilidade co algoritmo implementado, non
	obstante, inda que existen algunhas utilidades que facilitan o desenvolvemento web con 
	esta linguaxe como Wt (Web Toolkit)\cite{wt-url}, o grado de axilidade está moi por 
	baixo do que facilitan o resto das combinacións. Tamén pode resultar de interese o
	coñecido proxecto Icecast\cite{icecast-url}, que fai streaming de vídeo sobre unha 
	interface web.
	
	\subsection{Python + Django}
	Python presentase como a mellor opción para desenvolver o lado servidor, por unha parte
	dispón do módulo Subprocess\cite{subprocess-module-url} que permite executar calquera
	comando pola terminal maximizando así a modularidade e a integración co algoritmo en C++.
	Por outra parte Django\cite{django-web-page-url} contén un potente ORM e un sistema de 
	"Templates" que simplifica a parte web. Para concluír cabe destacar o feito de que 
	python sexa unha linguaxe interpretada, xa que isto evita o paso previo de compilación.
	
	
\section{Front-End}
	Unha vez seleccionada a tecnoloxía do lado servidor, é hora de ver que opcións existen para
	o Front-End, a parte do Sistema encargada da interacción co usuario. O maior requisito de esta 
	parte é ser capaz de representar todas as capas de información complexa que xera o
	Back-End. E tratándose dunha aplicación web estudaranse aqueles xeitos que permitan
	a reprodución de vídeo, e a maiores o debuxado de figuras en movemento sobre este vídeo.\\
	
	Destacan principalmente dúas alternativas:
	
	\subsection{Vídeo en Flash}
		Vídeo Flash é a tecnoloxía de reprodución de vídeo mais empregada e madura en
		internet dende hai anos. Inicialmente creada por Macromedia e mercada por Adobe 
		en 2005, permite crear elaboradas animacións vectoriais, que logo poden proxectarse
		sobre un vídeo, mentres que tamén manexa os eventos de reprodución de vídeo como o 
		play ou o stop.\\
		
		Ten certos problemas en tanto ao Posicionamento Web(SEO), reprodución en 
		dispositivos móbiles, accesibilidade... pero o meirande de todos eles é que mentres
		que outros dos exemplos estudados son 100\% libres, Flash é un programa propietario
		para o que é preciso adquirir unha licencia.		
		
	
	\subsection{Vídeo HTML5 + Javascript}
		Esta é outra das combinacións mais empregadas actualmente, xa que segue o estándar
		do W3C (World Wide Web Consortium)\cite{w3schools-video-tag}. Nel defínese como se
		han de mostrar e obter os vídeos dunha páxina codificada coa linguaxe HTML5, e 
		destaca por seres extremadamente sinxelo en comparación con Flash ou outras tecnoloxías.
		
		Para a modificación da súa aparencia pódese empregar \emph{CSS3} (Follas de Estilos en 
		Cascada), e para modificar o seu comportamento, o elemento $<video>$ de HTML5 achega
		unha serie de Métodos, Eventos e Propiedades\cite{w3school-video-events} que poden
		empregarse dende código \emph{Javascript}.
		
		Isto resulta de moita utilidade, xa que mediante Javascript tamén pode manexarse o
		elemento $<canvas>$ de HTML5, que xera un mapa de bits para construír gráficos, 
		manipular imaxes e crear dinamicamente animacións nunha páxina web. A única dúbida que
		soe xurdir sobre esta tecnoloxía está en canto ao seu rendemento e alcance, pero
		exemplos como os que amosa Kevin Roast na súa páxina web\cite{kevin-roast-canvas-examples}
		despexan toda dúbida posíbel. 
		
		Tamén é de interese a librería de estilos web \emph{Twiter Bootstrap}
		\cite{bootstrap-page-url} para axilizar o deseño web con CSS3, Así como a 
		librería de Javascript \emph{JQuery} \cite{jquery-page-url} que minimiza de 
		xeito considerable as liñas de código a implementar.